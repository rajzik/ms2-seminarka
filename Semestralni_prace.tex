%----------------------------------------------------------------------------------------
%	Konfigurace dokumentu
%----------------------------------------------------------------------------------------

\documentclass[paper=a4, fontsize=11pt]{scrartcl} 
\usepackage[T1]{fontenc} 
\usepackage{fourier} 
\usepackage[utf8]{inputenc}
\usepackage[main=czech, english]{babel}
\usepackage{amsmath,amsfonts,amsthm} 
\usepackage{lipsum}
\usepackage{sectsty}
\usepackage{multirow}
\usepackage{color}
\usepackage{float}
\usepackage{tabularx}
\allsectionsfont{\centering \normalfont\scshape}
\usepackage{fancyhdr} 
\pagestyle{fancyplain} 
\fancyhead{} 
\fancyfoot[L]{}
\fancyfoot[C]{} 
\fancyfoot[R]{\thepage}
\renewcommand{\headrulewidth}{0pt}
\renewcommand{\footrulewidth}{0pt}
\setlength{\headheight}{13.6pt} 
\numberwithin{equation}{section}
\numberwithin{figure}{section} 
\numberwithin{table}{section}
\setlength\parindent{0pt} 
\usepackage[ddmmyyyy]{datetime}
\newdate{date}{26}{09}{2017}
\date{\displaydate{date}}

\addto\captionsenglish{% Replace "english" with the language you use
\renewcommand{\contentsname}%
  {Obsah}%
}

%----------------------------------------------------------------------------------------
%	Titulek
%----------------------------------------------------------------------------------------

\newcommand{\horrule}[1]{\rule{\linewidth}{#1}} 
\title{	
\normalfont \normalsize 
\textsc{Mendelova univerzita v Brně\\Provozně ekonomická fakulta} \\ [25pt] 
\horrule{0.5pt} \\[0.4cm] 
\huge Modelování pokrytí terénu technologií Wi-Fi \\ % Název semestrální práce
\horrule{2pt} \\[0.5cm] 
}

\author{Jan Šilhan} % Vaše jméno

\date{\displaydate{date}} 

\begin{document}

\maketitle 

\newpage

\tableofcontents

\newpage

%----------------------------------------------------------------------------------------
%	Úvod
%----------------------------------------------------------------------------------------

\section{Úvod}
Wi-Fi je označení pro bezdrátový přenos dat. Pro přenos se používají mikrovlny, které se šíří v budovách a ve volných prostranstvích. Jelikož šíření signálu venku není velký úkol zaměříme se na pokrytí vnitřku budov. 
Pro modelování lze použít několik softwarových řešení kde lze namodelovat budovu s překážkami. Poté lze velmi jednoduše nasimulovat sílu signálu v různých umístění. 
Tato práce se nezabývá standarty IEEE 802.11 ani infrastruktury sítě. 



\section{Přenos signálu}
	\subsection{Útlum signálu v atmosféře}
	Je ztráta signálu v běžných podmínkách, bez překážek. 
	\\*	
	Pro který se používá se následující vztah:
	\begin{equation}
		L_{0} = 20 * log \left(
			\frac{4*\pi*d}
			{\alpha}
			\right)
	\end{equation}
	kde:
	\\*
	$L_0$ - Ztráta vlivem průchodu atmosférou; [dB]
	\\*
	$d$ - Vzdálenost mezi anténami; [m]
	\\*
	$\alpha$ - Vlnová délka; [m]
	\subsection{Útlum signálu s překážkami}
	Překážky jsou buď stacionární nebo pohyblivé. 
	Útlum lze zjistit měřením a následným výpočtem. 
	Se zvyšující se frekvencí klesá prostupnost. 
	Proto se používá maximálně 5GHz 
	(s výjimkou 60 GHz směrovače který je ve vývoji a vysoce experimentální).
	\\*
	Pro výpočet se používá následující vztah:
	\begin{equation}
	L = 10 * log \left(
		\frac{P_{2}}
		{P_{1}}
		\right)
	\end{equation}
	Kde:
	\\*
	$L$ - Ztráty vlivem překážek; [dB]
	\\*
	$P_{1}$ - Výkon vyslaný; [dB]
	\\*
	$P_{2}$ - Výkon přijatý; [dB]
	\newpage
	\subsection{Výkon AP}
		\subsubsection{Nařízení ČTÚ}
		\begin{table}[H]
		\centering
		\begin{tabularx}{\linewidth}{|X|X|X|X|} 
		\hline
		\textcolor[rgb]{0.2,0.2,0.2}{\textbf{Kmitočtové pásmo}}             & \textcolor[rgb]{0.2,0.2,0.2}{\textbf{Vyzářený výkon}}  & \textcolor[rgb]{0.2,0.2,0.2}{\textbf{Maximální spektrální hustota EIRP}}                             & \textcolor[rgb]{0.2,0.2,0.2}{\textbf{Další podmínky}}             \\ 
		\hline
		\multirow{2}{*}{\textcolor[rgb]{0.2,0.2,0.2}{2400,0 - 2483,5 MHz} } & \multirow{2}{*}{\textcolor[rgb]{0.2,0.2,0.2}{100 mW} } & \textcolor[rgb]{0.2,0.2,0.2}{10 mW/1 MHz}                                                            & \textcolor[rgb]{0.2,0.2,0.2}{systémy s technikou DSSS nebo OFDM}  \\ 
		
																		&                                                        & \textcolor[rgb]{0.2,0.2,0.2}{100 mW/100 kHz}                                                         & \textcolor[rgb]{0.2,0.2,0.2}{systémy s technikou FHSS}            \\ 
		\hline
		\textcolor[rgb]{0.2,0.2,0.2}{5150 - 5250 MHz}                       & \textcolor[rgb]{0.2,0.2,0.2}{200 mW střední}           & \textcolor[rgb]{0.2,0.2,0.2}{10 mW/MHz (střední spektrální hustota v libovolném úseku širokém 1MHz)} & \textcolor[rgb]{0.2,0.2,0.2}{pouze pro použití uvnitř budovy}     \\ 
		\hline
		\textcolor[rgb]{0.2,0.2,0.2}{5470 - 5725 MHz}                       & \textcolor[rgb]{0.2,0.2,0.2}{1 W střední}              & \textcolor[rgb]{0.2,0.2,0.2}{50 mW/MHz (střední spektrální hustota v libovolném úseku širokém 1MHz)} & -                                                                 \\
		\hline
		\end{tabularx}
		\end{table}



	\subsection{Modely pro plánování vnitřních prostor}
	Modely slouží pro predikci pokrytí uvnitř i vně. 
	Pomáhají rozmístit přístupové body, 
	nastavováním kanálů a výkonu.
	\subsubsection{One-Slope model}
	Empirický model, 
	pro velké městské buňky, 
	do dnes využívaný. 
	Není náročný na zadání vstupních dat a je rychlý na výpočet. 
	Využívá jeden útlum pro celou oblast, 
	udávají se útlumy pro různá prostředí a frekvenční pásma.
	\subsubsection{Multi-Wall model}
	Je daleko složitější a potřebuje přesné rozmístění příček na patrech i s jejich útlumem. 
	Ovšem tento model je úměrně složitý k výstupu. 
	Omezení tohoto modelu je že nelze simulovat vlnovodný efekt v dlouhých a zahnutých chodbách.
\newpage

\section{Softwarové řešení}
Řešení pro simulaci šíření není mnoho, 
jeden z nich je I-prop který však potřebuje přídavný hardware pro měření. 
Návrh rozložení přístupových bodů je stále složitý, 
ale tento software pomůže jak s rozložením, 
tak rozvrhnutím kanálů.

\section{Závěr}
Modelování pokrytí terénu není jednoduchá disciplína, 
je zde neuvěřitelné množství překážek se kterými je nutno se vypořádat. 
Venkovní pokrytí je jednoduší z několika důvodů, 
například daleko méně překážek, 
ale jsou zde jiné těžkosti. 
Jelikož města a celkově česká republika nemá regulované pásmo 
a poskytovatelé si mohou vybírat kanály které obsadí je zde velké rušení, 
Brno je toho velkým příkladem. 
Model uvnitř budov je komplexnější a je zde mnoho možností pochybení. 
Které může začínat výběrem špatného zařízení a končit slepými místy bez signálu. 
Simulace nejsou přesné a vždy je nutná korektura v reálu.
\end{document}